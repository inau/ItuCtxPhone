\documentclass{scrartcl}
\usepackage{graphicx,hyperref,amsmath,natbib,bm,url, xstring}
\usepackage{microtype,todonotes}
\usepackage[australian]{babel}
\usepackage[a4paper,text={16.5cm,25.2cm},centering]{geometry}
\usepackage[compact,small]{titlesec}
\usepackage{listings}
\usepackage{pdfpages}
\setlength{\parskip}{1.2ex}
\setlength{\parindent}{0em}
\clubpenalty = 10000
\widowpenalty = 10000
\usepackage{kpfonts}
%\usepackage[T1]{fontenc}

\usepackage{color}
\definecolor{mygreen}{rgb}{0,0.6,0}
\definecolor{mygray}{rgb}{0.5,0.5,0.5}
\definecolor{mymauve}{rgb}{0.58,0,0.82}

%Code block style
\lstset{ %
  backgroundcolor=\color{white},   % choose the background color; you must add \usepackage{color} or \usepackage{xcolor}
  basicstyle=\footnotesize,        % the size of the fonts that are used for the code
  breakatwhitespace=false,         % sets if automatic breaks should only happen at whitespace
  breaklines=true,                 % sets automatic line breaking
  captionpos=b,                    % sets the caption-position to bottom
  commentstyle=\color{mygreen},    % comment style
  deletekeywords={...},            % if you want to delete keywords from the given language
  escapeinside={\%*}{*)},          % if you want to add LaTeX within your code
  extendedchars=true,              % lets you use non-ASCII characters; for 8-bits encodings only, does not work with UTF-8
  frame=lines,	                   % adds a frame around the code
  keepspaces=true,                 % keeps spaces in text, useful for keeping indentation of code (possibly needs columns=flexible)
  keywordstyle=\color{blue},       % keyword style
  language=Octave,                 % the language of the code
  otherkeywords={*,...},           % if you want to add more keywords to the set
  numbers=left,                    % where to put the line-numbers; possible values are (none, left, right)
  numbersep=10pt,                   % how far the line-numbers are from the code
  numberstyle=\tiny\color{black}, % the style that is used for the line-numbers
  rulecolor=\color{black},         % if not set, the frame-color may be changed on line-breaks within not-black text (e.g. comments (green here))
  showspaces=false,                % show spaces everywhere adding particular underscores; it overrides 'showstringspaces'
  showstringspaces=false,          % underline spaces within strings only
  showtabs=false,                  % show tabs within strings adding particular underscores
  stepnumber=1,                    % the step between two line-numbers. If it's 1, each line will be numbered
  stringstyle=\color{mymauve},     % string literal style
  tabsize=2,	                   % sets default tabsize to 2 spaces
  title=\lstname                   % show the filename of files included with \lstinputlisting; also try caption instead of title
}

%Code functions
\newcommand{\code}[2][c]{ \lstinputlisting[language=Java, caption= #2 ]{#2}}
\newcommand{\codeRange}[4][c]{\lstinputlisting[language=Java, caption= #2, firstnumber = #3, firstline=#3, lastline=#4]{#2}}


\begin{document}
\title{Pervasive Project}
\subtitle{Cloud infrastructure}
\author{Ivan Naumovski, inau@itu.dk}

\maketitle
\tableofcontents

\section{The Cloud}
We are using the Google App Engine to host our cloud, and we use their DataStorage API for persistance(Objectify is the ORM framework we use).

We had a thorough discussion on how exactly beacons and context information should be linked together, we settled on having two tables with no foreign key mappings.\\
One table for beacon data and one table for context information.\\

The beacon table is as follows:
\begin{quotation}
\textbf{Beacon}(\underline{key}, uid, major, minor, lat, lng, updated);
\end{quotation}
The key is generated as a combination of the uid, major and minor values, even though there is some redundancy when doing so.

The context table is as follows:
\begin{quotation}
\textbf{Context}(\underline{uid}, type, values, lat, lng, updated);
\end{quotation}
The idea is that a single location can have multiple contexts. This is due to the fact that the sensors we have chosen make more sense when we can track values over time (sound and pressure).
This would not have been possible if only one context existed per beacon.

The idea is that we select contexts for a specific location and present their change in values over time or do some notifications based on the change in values.

The approach we went with is using the sensed beacons(not to be mistaken with the stored beacon entities) to pinpoint the closest beacon.

Once the closest beacon has been detected, we look it up in our local cache - if the beacon is not in the local SQLite instance, we look for it in the cloud.

In case it doesnt exist in the cloud, we create one using the cloud API, this will respond with the freshly created entity as a result, which is then stored in the local SQLite instance.

We have built two java servlets supporting CRUD functionality for both beacons and contexts.

\end{document}